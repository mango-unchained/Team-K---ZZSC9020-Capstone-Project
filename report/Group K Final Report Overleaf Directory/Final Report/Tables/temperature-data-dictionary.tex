\begin{table}[ht]
\centering
\setlength{\tabcolsep}{4pt} % Adjust the space between columns
\renewcommand{\arraystretch}{1.5}  % Increase the space between rows
\small % or \scriptsize or \footnotesize to reduce the font size
\begin{tabular}{@{}lp{2.3cm}llp{3.2cm}@{}} % Adjust column widths
\toprule
Field Name & Description & Data Type & Example & Statistics \\
\midrule
DATETIME & The UTC date-time of the demand recording in 30 min increments & DATETIME & 2010-01-01 00:00:00& 
\textbullet\ Maximum: 2021-03-18 00:00:00\newline
\textbullet\ Minimum: 2010-01-01 00:00:00\newline
\textbullet\ NULL Count: 0 \\

TEMPERATURE & The recorded temperature for that location & Float & 5561.21 & 
\textbullet\ Maximum: 44.7\newline
\textbullet\ Minimum: -1.3\newline
\textbullet\ Mean: 18.7\newline
\textbullet\ Std. Dev.: 6.0\newline
\textbullet\ NULL Count: 0 \\

state & The Australian state the reading corresponds to & \parbox[t]{2.3cm}{Enumerated String:\newline
\textbullet\ QLD\newline
\textbullet\ NSW\newline
\textbullet\ VIC\newline
\textbullet\ SA} & QLD &
\textbullet\ NULL Count: 0 \\

LOCATION & Location description of the reading & \parbox[t]{2.3cm}{Enumerated String:\newline
\textbullet\ Brisbane Archerfield Airport\newline
\textbullet\ Bankstown\newline
\textbullet\ Melbourne (Olympic Park)\newline
\textbullet\ Adelaide (Kent Town)} & Bankstown &
\textbullet\ NULL Count: 0 \\

\bottomrule
\end{tabular}
\caption{Data dictionary and summary statistics for temperature dataset}
\label{metadata}
\end{table}